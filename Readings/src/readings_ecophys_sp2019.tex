\documentclass[12pt, notitlepage]{article}   	% use "amsart" instead of "article" for AMSLaTeX format
\usepackage{geometry}                		% See geometry.pdf to learn the layout options. There are lots.
\geometry{a4paper}                   		% ... or a4paper or a5paper or ... 
%\geometry{landscape}                		% Activate for rotated page geometry
\usepackage[parfill]{parskip}    		% Activate to begin paragraphs with an empty line rather than an indent
\usepackage{graphicx}				% Use pdf, png, jpg, or eps§ with pdflatex; use eps in DVI mode
								% TeX will automatically convert eps --> pdf in pdflatex

\usepackage{hyperref}


%SetFonts

\usepackage[T1]{fontenc}
\usepackage[utf8]{inputenc}

\usepackage{tgbonum}

%SetFonts

\title{
	\textbf{
		Readings
	} \\
	\large Plant Physiological Ecology \\
	\large Spring 2019
}

\date{\vspace{-5ex}}

\begin{document}

{\fontfamily{phv}\selectfont %select helvetica (code = phv)

\maketitle

**Please contact Dr. Smith if you have trouble accessing the articles**

**Note: this file will be updated to account for changes to the schedule**

\section*{Week of January 21}
\textit{Classical Literature Tuesday - Jan 22} \par
Chapin FS. 2003. Effects of Plant Traits on Ecosystem and Regional Processes: 
a Conceptual Framework for Predicting the Consequences of Global Change. 
Annals of Botany 91: 455–463. \par
\url{https://academic.oup.com/aob/article/91/4/455/213070}

\textit{Recent Literature Thursday - Jan 24} \par
Reich PB. 2014. The world-wide ‘fast–slow’ plant economics spectrum: a traits manifesto. 
Journal of Ecology 102: 275–301. \par
\url{https://besjournals.onlinelibrary.wiley.com/doi/10.1111/1365-2745.12211}

\section*{Week of January 28}
\textit{Classical Literature Tuesday - Jan 29} \par
Von Caemmerer S, Farquhar GD. 1981. Some relationships between the biochemistry of 
photosynthesis and the gas exchange of leaves. Planta 153: 376–387. \par
\url{https://link.springer.com/article/10.1007/bf00384257}

\textit{Recent Literature Thursday - Jan 31} \par
Smith NG, Dukes JS. 2018. Drivers of leaf carbon exchange capacity across biomes at 
the continental scale. Ecology 99: 1610–1620. \par
\url{https://esajournals.onlinelibrary.wiley.com/doi/full/10.1002/ecy.2370}

\section*{Week of February 4}
\textit{Classical Literature Tuesday - Feb 5} \par
Boardman NK. 1977. Comparative photosynthesis of sun and shade plants. 
Annual review of plant physiology 28: 355–377. \par
\url{https://www.annualreviews.org/doi/10.1146/annurev.pp.28.060177.002035}

\textit{Recent Literature Thursday - Feb 7} \par
Niinemets Ü, et al. 2015. A worldwide analysis of within-canopy variations in leaf 
structural, chemical and physiological traits across plant functional types.
New Phytologist 205: 973–993. \par
\url{https://nph.onlinelibrary.wiley.com/doi/full/10.1111/nph.13096}

\section*{Week of February 11}
\textit{Classical Literature Tuesday - Feb 12} \par
Atkin OK and Tjoelker M. 2003. Thermal acclimation and the dynamic response of plant 
respiration to temperature. Trends in Plant Science 8: 343–351. \par
\url{https://www.sciencedirect.com/science/article/pii/S1360138503001365}

\textit{Recent Literature Thursday - Feb 14} \par
Saenz et al. 2018. In situ warming in the Antarctic: effects on growth and photosynthesis 
in Antarctic vascular plants. New Phytologist 218: 1406-1418. \par
\url{https://nph.onlinelibrary.wiley.com/doi/full/10.1111/nph.15124}

\section*{Week of February 18}
\textit{Classical Literature Tuesday - Feb 19} \par
Chaves MM, Pereira JS, Maroco J, et al. 2002. How Plants Cope with Water Stress 
in the Field? Photosynthesis and Growth. Annals of Botany 89: 907–916. \par
\url{https://academic.oup.com/aob/article/89/7/907/151103}

\textit{Recent Literature Thursday - Feb 21} \par
TBD \par
%\url{https://onlinelibrary.wiley.com/doi/full/10.1111/j.1365-2486.2012.02797.x}

\section*{Week of February 25}
\textit{Classical Literature Tuesday - Feb 26} \par
Bazzaz FA. 1990. The response of natural ecosystems to the rising global CO2 levels. 
Annual review of ecology and systematics 21: 167–196. \par
\url{https://www.annualreviews.org/doi/10.1146/annurev.es.21.110190.001123}

\textit{Recent Literature Thursday - Feb 28} \par
TBD \par
%\url{https://onlinelibrary.wiley.com/doi/full/10.1111/j.1365-2486.2012.02797.x}

\section*{Week of March 4}
\textit{Classical Literature Tuesday - Mar 5} \par
Brix H. 1971. Effects of nitrogen fertilization on photosynthesis and respiration in 
Douglas-fir. Forest Science 17: 407–414. \par
\url{https://academic.oup.com/forestscience/article-abstract/17/4/407/4709889}

\textit{Recent Literature Thursday - Mar 7} \par
TBD \par
%\url{https://onlinelibrary.wiley.com/doi/full/10.1111/j.1365-2486.2012.02797.x}


\section*{Week of March 18}
\textit{Classical Literature Tuesday - Mar 19} \par
Mooney HA. 1972. The carbon balance of plants. 
Annual review of ecology and systematics 3: 315–346. \par
\url{https://www.annualreviews.org/doi/10.1146/annurev.es.03.110172.001531}

\textit{Recent Literature Thursday - Mar 7} \par
TBD \par
%\url{https://onlinelibrary.wiley.com/doi/full/10.1111/j.1365-2486.2012.02797.x}

\section*{Week of March 25}
\textit{Classical Literature Tuesday - Mar 26} \par
Givnish TJ. 2002. Adaptive significance of evergreen vs. deciduous leaves: 
solving the triple paradox. Silva fennica 36: 703–743. \par
\url{https://silvafennica.fi/article/535}

\textit{Recent Literature Thursday - Mar 28} \par
TBD \par
%\url{https://onlinelibrary.wiley.com/doi/full/10.1111/j.1365-2486.2012.02797.x}

\section*{Week of April 8}
\textit{Classical Literature Tuesday - Apr 9} \par
Grime JP. 1977. Evidence for the Existence of Three Primary Strategies in Plants and Its 
Relevance to Ecological and Evolutionary Theory. 
The American Naturalist 111: 1169–1194. \par
\url{https://www.jstor.org/stable/2460262}

\textit{Recent Literature Thursday - Apr 11} \par
TBD \par
%\url{https://onlinelibrary.wiley.com/doi/full/10.1111/j.1365-2486.2012.02797.x}

\section*{Week of April 15}
\textit{Classical Literature Tuesday - Apr 16} \par
Wright DP, Scholes JD, Read DJ. 1998. Effects of VA mycorrhizal colonization on 
photosynthesis and biomass production of Trifolium repens L. 
Plant, Cell and Environment 21: 209–216. \par
\url{https://onlinelibrary.wiley.com/doi/10.1046/j.1365-3040.1998.00280.x}

\textit{Recent Literature Thursday - Apr 18} \par
TBD \par
%\url{https://onlinelibrary.wiley.com/doi/full/10.1111/j.1365-2486.2012.02797.x}

\section*{Week of April 22}
\textit{Classical Literature Tuesday - Apr 23} \par
Aerts R. 1997. Climate, Leaf Litter Chemistry and Leaf Litter Decomposition in 
Terrestrial Ecosystems: A Triangular Relationship. Oikos 79: 439–449. \par
\url{https://www.jstor.org/stable/3546886}

\textit{Recent Literature Thursday - Apr 25} \par
TBD \par
%\url{https://onlinelibrary.wiley.com/doi/full/10.1111/j.1365-2486.2012.02797.x}

\section*{Week of April 29}
\textit{Classical Literature Tuesday - Apr 30} \par
Field CB, Lobell DB, Peters HA, Chiariello NR. 2007. Feedbacks of Terrestrial Ecosystems 
to Climate Change. Annual Review of Environment and Resources 32: 1–29. \par
\url{https://www.annualreviews.org/doi/10.1146/annurev.energy.32.053006.141119}

\textit{Recent Literature Thursday - May 2} \par
TBD \par
%\url{https://onlinelibrary.wiley.com/doi/full/10.1111/j.1365-2486.2012.02797.x}

} %end font selection

\end{document}