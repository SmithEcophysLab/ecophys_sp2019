\documentclass[12pt, notitlepage]{article}   	% use "amsart" instead of "article" for AMSLaTeX format
\usepackage{geometry}                		% See geometry.pdf to learn the layout options. There are lots.
\geometry{a4paper}                   		% ... or a4paper or a5paper or ... 
%\geometry{landscape}                		% Activate for rotated page geometry
\usepackage[parfill]{parskip}    		% Activate to begin paragraphs with an empty line rather than an indent
\usepackage{graphicx}				% Use pdf, png, jpg, or eps§ with pdflatex; use eps in DVI mode
								% TeX will automatically convert eps --> pdf in pdflatex

\usepackage{hyperref}


%SetFonts

\usepackage[T1]{fontenc}
\usepackage[utf8]{inputenc}

\usepackage{tgbonum}

%SetFonts

\title{
	\textbf{
		Mini-Quiz 1
	} \\
	\large BIOL 4350/6350 \\
	\large Ecophysiology \\
}

\date{\vspace{-5ex}}

\def\wl{\par \vspace{\baselineskip}}

\begin{document}

{\fontfamily{phv}\selectfont %select helvetica (code = phv)

\maketitle

\section{\small{How does physiological ecology fit into the 
hierarchy of ecological studies?}}
\wl
\wl
\wl
\wl
\wl
\wl
\wl
\wl
\wl
\wl
\wl
\wl
\wl
\wl

\section{\small{Write three examples of how plant ecophysiology affects your life.}}

\newpage

\section{\small{Chapin mentioned that we should consider plant physiological impacts on
large scale processes (e.g., community and ecosystem processes) through the lens of
biodiversity. What do you think he meant by that?}}

\wl
\wl
\wl
\wl
\wl
\wl
\wl
\wl
\wl
\wl
\wl
\wl
\wl
\wl

\section{\small{Why might plant ecophysiology be important for predicting and preparing for
the future effects of anthropogenic global change?}}

} %end font selection

\end{document}